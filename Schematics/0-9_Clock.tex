\documentclass[tikz,border=2pt]{standalone}
\usepackage[european]{circuitikz}
\usepackage{siunitx}

\usetikzlibrary{backgrounds}

\begin{document}
\begin{circuitikz}[scale=1.5, transform shape, thick, font=\Large]

    % Manual bounding box setting
    \draw[use as bounding box] (-11.0,-1) rectangle (13.5,7.0);

    % To see it on github
    \begin{scope}[on background layer]
        \fill[white] (-11.0,-1) rectangle (13.5,7.0);
    \end{scope}



    % #1 - x shift of block
    % #2 - 0 - only LED, 1 - LED + resistor + pins
    % #3 - list of x/num pairs for LEDs, e.g. 0/16, 1/15 
    \newcommand{\LedBlock}[3]{
        
        % Wire going up with open dot and 5V label
        \ifnum#2=1
            \draw ({#1-10},5) 
            to[short,-o] ({#1-10},5.5) node [right]{14} to ({#1-10},5.8) node[vcc]{+\SI{5}{\volt}};
        \else
            \draw ({#1-10},5) 
            to[short, -o] ({#1-10},5.5) node [right]{13};
        \fi

        % Common anode
        \draw ({#1-10},5) -- ({#1-10+7*1.5},5);

        \foreach \x/\num in {#3} {
            % LEDs
            \draw
                ({#1-10+\x*1.5},5)
                to[led, o-o] ({#1-10+\x*1.5},3.5) node[right]{\num};

            % Resistors + pins
            \ifnum#2=1
                \draw
                    ({#1-10+\x*1.5},3.5)
                    to[R, o-*] ({#1-10+\x*1.5},1)
                    node[below]{PD\x};
            \fi
        }
        

    }
    

    \LedBlock{0}{1}{0/16,1/15,2/3,3/2,4/1,5/18,6/17,7/4}

    \LedBlock{12}{0}{0/11,1/10,2/8,3/6,4/5,5/12,6/7,7/9}



\end{circuitikz}
\end{document}
